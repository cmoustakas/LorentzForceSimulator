\documentclass{article}

\usepackage{amsmath}
\usepackage{graphicx}
\usepackage{amssymb}
\usepackage{float}
\usepackage{tikz}
\usepackage{fancyhdr}

\pagestyle{fancy}


\fancyhead{}
\fancyhead[RO,LE]{Lorentz Force Simulator}
\fancyfoot{}
\fancyfoot[LE,RO]{\thepage}
\fancyfoot[LO,CE]{Section \thesection}
\fancyfoot[CO,RE]{Chares Moustakas}

\usetikzlibrary{shapes.geometric, arrows}

\title{
	{
	Lorentz Force Simulator \\
	\vspace{3cm}
	} \\
	
}
\author{Chares Moustakas}
\date{15/12/2022}



\begin{document}

	\maketitle




\section{Introduction}
	\paragraph{}
	The current software is attempting to simulate the behaviour of a particle near a finite long charged wire. It is important to underline that the aforementioned software is just a fun, non-optimized project and as a consequence the reader can feel free to use, modify, improve and play with it. It is notable to mention that this software was developed in C++ using relatively low level APIs and Libraries such as modern OpenGL, custom fragment-vertex shaders through GLSL, Dear ImGUI and CMake, while all the Objects-meshes were constructed through Blender.     

\section{Mathematics}
	\paragraph{}
	The concept of this simulation is to calculate the Lorentz Force that the wire exterts to the spherical particle, using primarily the current that flows through the wire as a datum. To begin with, it is crucial to visualize from the Mathematical perspective the goal, being said that the formula for the precious calculation is:  
			\begin{equation*}
			\overrightarrow{\rm F_{L}} = q(\overrightarrow{\rm E} + \overrightarrow{\rm u} \times \overrightarrow{\rm B})
			\end{equation*}
	
	\paragraph{}
	Where q is Coulomb charge of the particle, \textbf{E} the electric field, \textbf{u} particle's velocity and \textbf{B} the magnetic's field flux density. 
	\subsection{B calculation}
	\paragraph{}
	The calculation of \textbf{B} is based on Ampere's formula:
			\begin{equation*}
			\oint_C \overrightarrow{\rm B}\overrightarrow{\rm dl} = \mu I
			\end{equation*}
	Due to \textbf{B} and \textbf{dl} parallelism, the equation above is tranformed as:
			\paragraph{}
			\begin{equation*}
			\overrightarrow{\rm |B|} \oint_C dl &= \mu I \\
			\end{equation*}
			
			\begin{equation*}
			\overrightarrow{\rm |B|} 2\pi r &= \mu I \\
			\end{equation*}
			
			\begin{equation*}
			\overrightarrow{\rm |B|} &= \dfrac{\mu I}{2\pi r}  \\
			\end{equation*}
			
			\begin{equation*}
			\overrightarrow{\rm B} &= \dfrac{\mu I}{2\pi r} \hat{\phi} 
			\end{equation*}
			
			
			Cylindrical Coordinates:
			\begin{equation*}
			\hat{\phi} = -\dfrac{z}{r}\hat{x} + \dfrac{x}{r}\hat{z}  
			\end{equation*}
			
	It is notable that the direction of \textbf{B} is tangent to the closed circle-loop C. Consequently \textbf{B} calculation is possible if the position of the particle and the current are known magnitudes.
	
	\subsection{E calculation}
	\paragraph{}
	Regarding Electric field calculation, patience is the key... So lets construct the main acceptance, first of all lets assume that the wire has \textbf{L} length. An infinitesmal charged part of the wire \textbf{dq} that is \textbf{y} units away from the end of the cable causes  Electric field \textbf{dE} for a random point P that it is \textbf{r} units away from the wire:
		    
		    \begin{figure}[H]
  			\includegraphics[width=\linewidth]{/home/robin/Pictures/lorentz.png}
  			\caption{ Calculation of Electric field}
			\centering
		    \end{figure}
		    
		    
		    \begin{align*}
		    \dfrac{dq}{dy} &= \dfrac{Q}{L} \\
		    dq &= \dfrac{Q}{L} dy
		    \end{align*}
		    
		    \begin{equation} 
		    \overrightarrow{\rm dE} &= dE_y sin\theta \hat{y} + dE_r cos\theta \hat{r}
		    \end{equation}
		    
		    \begin{equation}
		    sin\theta = \dfrac{y}{\sqrt{y^2 + r^2}},
		    cos\theta = \dfrac{r}{\sqrt{y^2 + r^2}}
		    \end{equation}
		    
		    \begin{equation*} 
		    dE &= k \dfrac{dq}{y^2 + r^2}
		    \end{equation*}\vspace{1cm}
		    Based on the above:
		    \begin{equation*} 
		    dE_y &= k \dfrac{Q}{L}\dfrac{y}{\sqrt{(y^2 + r^2)^3}}dy
		    \end{equation*}
		    
		    \begin{equation*}
		    \int_{0}^{E_y} dE_y = k \dfrac{Q}{L} \int_{-L/2}^{L/2} \dfrac{y}{\sqrt{(y^2 + r^2)^3}} dy
		    = k\dfrac{Q}{L}( \dfrac{-1}{\sqrt{(r^2 + \dfrac{L^2}{4})}} + \dfrac{1}{\sqrt{(r^2 + \dfrac{L^2}{4})}} )
		    \end{equation*} 
		    
		    \begin{equation*}
		    E_y = 0
		    \end{equation*}
		    The only non zero coordinate is the coordinate of r unit vector, based on the (1) ! Lets calculate the r coordinate.
		    
		    \begin{equation*}
		     dE_r &= k \dfrac{Q}{L}\dfrac{r}{\sqrt{(y^2 + r^2)^3}} dy
		    \end{equation*}
		     
		    \begin{equation*}
		    \int_{0}^{E_r} dE_r = k \dfrac{Q}{L} \int_{-L/2}^{L/2} \dfrac{r}{\sqrt{(y^2 + r^2)^3}} dy
		    = k \dfrac{Q}{L} ( \dfrac{L/2}{r \sqrt{(r^2 + \dfrac{L^2}{4})}} + \dfrac{L/2}{r \sqrt{(r^2 + \dfrac{L^2}{4})}})
		    \end{equation*}
		    
		    \begin{equation*}
		    E_r = k \dfrac{Q}{r\sqrt{r^2 + \dfrac{L^2}{4}}}
		    \end{equation*}
		    
		    \paragraph{}
		    Consequently, the expression (1) can be written as:

		    \begin{equation}
		    \overrightarrow{\rm E} =  k \dfrac{Q}{r\sqrt{r^2 + \dfrac{L^2}{4}}} \hat{r}
		    \end{equation}		    
		    
		    Cylindrical coordinates:
		    \begin{equation*}
		    \hat{r} = \dfrac{x}{r} + \dfrac{z}{r}
		    \end{equation*}
		    
		    
		    \paragraph{}
		    It is crucial to keep in mind that the Coulomb charge \textbf{Q} is not constant in time, therefore on each OpenGL iteration the expression (3), concerning the Electric field calculation, is taken into account: 
		    \begin{equation*}
		    I = \dfrac{dQ}{dt}
		    \end{equation*}
		    
		    
		    
\end{document}
